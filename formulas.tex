
% 整页 =================================
\left\|s_{m}-s_{n}\right\|=\left\|x_{n+1}+\cdots+x_{m}\right\| \leq \sum_{k=n+1}^{\infty}\left\|x_{k}\right\| \leq \sum_{k=n+1}^{\infty} \frac{1}{2^{n+1}}

故  \left\{s_{n}\right\}  是  E  中的基本列, 由  E  是巴拿赫空间,  s_{n}=\sum_{1=1}^{\infty} x_{k}  在  E  中收签.
令  x=\sum_{n=1}^{\infty} x_{n},  则

\|x\| \leq \sum_{n=1}^{\infty}\left\|x_{n}\right\| \leq \sum_{n=1}^{\infty} \frac{1}{2^{n}}=1, \text { 即有 } x \in O_{1} .

因为T连给, 在(4)中令  n \rightarrow \infty,  有  y=T x .  故  M_{1}=T\left(O_{1}\right) \supset Q_{\frac{\delta_{0}}{2}} .

% --------------------------------------


% 整页 =================================
Step 3:  证明  M_{1}=T\left(O_{1}\right) \text { 在 } Q_{\delta_{0}} 中稠密.

a) 因  E=\bigcup_{n=1}^{\infty} O_{n},  故  F=\bigcup_{n=1}^{\infty} M_{n} .  由于  F  是第二类型的集, 故\exists  n_{0}, \mathrm{s.t.}   M_{n_{0}}  不是稀疏集.

于是ヨ  E_{1}  中的闭球  Q\left(y_{0}, r_{0}\right)=:\left\{y \in E_{1}:\left\|y-y_{0}\right\| \leq r_{0}\right\}  使得  M_{n_{0}} 
在  Q\left(y_{0}, r_{0}\right)  中稠密.
% --------------------------------------


% 整页 =================================
b)  M_{n_{0}}  在  Q\left(y_{0}, r_{0}\right)  中稠密,  \Rightarrow M_{n_{0}}  在  Q\left(-y_{0}, r_{0}\right)  中稠密,
 \Rightarrow M_{n_{0}}  在  Q_{r_{0}}  中稠密,
 \Rightarrow M_{1}  在
稠密.
记:  \quad \delta_{0}=: \frac{r_{0}}{n_{0}} .  OK  ! 
注:  E_{1}  的完备性体现在何处? (第二型的赋范线性空间即可  ) 

% --------------------------------------


% 整页 =================================
推论  \quad T  满足定理2.1的条件\Rightarrow  T  将  E  中的开集映成  E_{1}  中的开集.
证  \quad  由定理证明可知,  T\left(O_{1}\right) \supset Q_{\frac{\delta_{0}}{2}} , 所以

\forall n, \quad T\left(O_{\frac{1}{2^{n}}}\right) \supset Q_{\frac{\delta_{0}}{2^{n+1}}}

设  G \subset E  是开集,  \forall y \in T(G), \exists x \in G,  s.t.  T x=y .
因为  x  是G的内点, 故存在  n  使得  x+O_{\frac{1}{2^{n}}} \subset G .
于是  T x+T\left(O_{\frac{1}{2} n}\right) \subset T(G),  即  y+T\left(O_{\frac{1}{2} n}\right) \subset T(G) . 
再注意到  Q_{\frac{\delta_{0}}{2^{n+1}}} \subset T\left(O_{\frac{1}{2^{n}}}\right) .  因此

y+Q \frac{\delta_{0}}{2^{n+1}}=Q\left(y, \frac{\delta_{0}}{2^{n+1}}\right) \subset T(G)

故  y  是  T(G)  的内点, 因此  T(G)  是开集.

% --------------------------------------


% 整页 =================================
推论2  \quad  设有界线性算子  T  将巴拿赫空间  E  映入巴拿赫空间  E_{1},  则  T  的值域或者是  E_{1}  或是  E_{1}  中第一类型的集, 二者必居其一.
注: 正是由于推论1中  T  将开集映成开集这一结论, 我们常将定理2.1称
为开映射定理.

 
% --------------------------------------


% 整页 =================================
当 T 的逆算子存在时 , 称 T 是可逆的

下面的定理给出了 T 有有界逆算子的一个重要条件 .
定理 2.2( 逆算子定理 ) 设有界线性算子 T 将巴拿赫空间 E 映成巴拿
赫空间 E_1 中的某个第二类型的集 , 而且 T 是单映射 , 则 T 存在有界逆算子 
% --------------------------------------


% 整页 =================================
证:由定理2.1,  T  的值域是  E .  再由假设,  T  是单映射. 故  T  的逆算
子  T^{-1}  存在.
记  x=T^{-1} y .  由不等式  (1),  有

\left\|T^{-1} y\right\|=\|x\| \leq K\|T x\|=K\|y\|

故  T^{-1}  有界.

% --------------------------------------


% 整页 =================================
定义 设  E  为线性空间,  \|\cdot\|_{1} 、\|\cdot\|_{2}  是定义在  E  上的两个范数,  E  按照这 两个范数均为赋范线性空间.
若存在正数  K_{1}, K_{2},  成立

K_{1}\|x\|_{1} \leq\|x\|_{2} \leq K_{2}\|x\|_{1}, \forall x \in E

则称范数 ||  \cdot \|_{1}  与  \|\cdot\|_{2}  等价.
将  E  分别赋以范数  \|\cdot\|_{1},\|\cdot\|_{2}  后得到的赋范线性空间分记为  \left(E,\|\cdot\|_{1}\right) ,  \left(E,\|\cdot\|_{2}\right) 
容易看出, 当  \|\cdot\|_{1},\|\cdot\|_{2}  等价时,  \left(E,\|\cdot\|_{1}\right),\left(E,\|\cdot\|_{2}\right)  拓扑同构, 反之, 亦
然.

% --------------------------------------


% 整页 =================================
推论  \quad  设  \left(E,\|\cdot\|_{1}\right),\left(E,\|\cdot\|_{2}\right)  均为巴拿赫空间,
若存在正数  K  使得对一切  x \in E,  有

\|x\|_{2} \leq K\|x\|_{1}

则  \|\cdot\|_{1}  与  \|\cdot\|_{2}  等价. 因此  \left(E,\|\cdot\|_{1}\right)  与  \left(E,\|\cdot\|_{2}\right)  拓扑同构.
证  \quad  令  I  是  E  上的单位算子, 则  I  可以看成是由巴拿赫空间  \left(E,\|\cdot\|_{1}\right)  到 巴拿赫空间  \left(E,\|\cdot\|_{2}\right)  上的线性算子, 且  I  显然是双映射.
由不等式(7) 可知,  I  还是有界的. 由定理2.2,  I  的逆算子  I^{-1}  也有界. 由
此可知  \|\cdot\|_{1},\|\cdot\|_{2}  等价.

% --------------------------------------


% 整页 =================================
设  E, E_{1}  为赋范线性空间, 作直和  E \oplus E_{1} , 在其中定义范数:

\|(x, y)\|=\|x\|+\|y\|, \quad \forall(x, y) \in E \oplus E_{1}

第七章 1 中已经指出,  E \oplus E_{1}  按照范数(8)是一个赋范线性空间.
定义2.1  \quad  设  T  是定义在  E  的子空间  D  上, 且值域包含在  E_{1}  中的线性算
子. 空间  E \oplus E_{1}  中的子集

G_{T} \triangleq\{(x, T x): \quad x \in D\}

称为T的图像.
对任何线性算子  T, G_{T}  是  E \oplus E_{1}  中的一个子空间.
注意到  G_{T}  在  E \oplus E_{1}  中可能是闭的, 也可能是非闭的. 如果  T  的图像  G_{T} 
是  E \oplus E_{1}  中的闭子空间, 则称  T  为闭线性算子或简称闭算子.

注:若 T 可逆 , 则 T 是闭算子 \Leftrightarrow T^{−1} 是闭算子
% --------------------------------------


% 整页 =================================
由于一般的线性算子不一定有界, 图像就成为研究这类算子的重要工
具. 下面的定理为线性算子成为闭算子提供一个易于检验的等价条件.
定理2.3  \quad  设  E, E_{1}  都是赋范线性空间.  T  是由  E  的子空间  D  到  E_{1}  中的
线性算子.
 T  为闭算子  \Longleftrightarrow  对  \forall\left\{x_{n}\right\} \subset D,  若  x_{n} \rightarrow x  in  E, T x_{n} \rightarrow y  in  E_{1}, 
则  x \in D  且  T x=y 
证  \Leftarrow \forall(x, y) \in \overline{G_{T}},  存在  \left\{x_{n}\right\} \in D  使得  \left(x_{n}, T x_{n}\right) \rightarrow(x, y) .
于是  \left\{x_{n}\right\} \rightarrow x,\left\{T x_{n}\right\} \rightarrow y .  由假设知,  x \in D, T x=y 
故  (x, y)=(x, T x) \in G_{T},  因此  G_{T}=\overline{G_{T}}, T  为闭算子.

% --------------------------------------


% 整页 =================================
\Rightarrow \quad  设  \left\{x_{n}\right\} \subset D,  且  x_{n} \rightarrow x, T x_{n} \rightarrow y,  这里  x \in E, y \in E_{1} .  于是

\left\{\left(x_{n}, T x_{n}\right)\right\} \rightarrow(x, y)

由  G_{T}  为  E \oplus E_{1}  中的闭集, 知  (x, y) \in G_{T} . 因此  x \in D,  且  T x=y .

对于一个给定的线性算子 , 现在已有连续性、有界性及闭性 . 因连续与
有界性等价 , 故本质上只有两个:有界性与闭性 . 有界性与闭性既有区
别又有联系 . 有界线性算子不一定是闭算子 , 闭算子也不一定有界 
% --------------------------------------


% 整页 =================================
定理2.4(闭图像定理)  \quad  设  T  是由巴拿赫空间  E  到巴拿赫空间  E_{1}  中 的线性算子.
 T  为有界算子  \Longleftrightarrow T  为闭算子.
证  \quad " \Leftarrow "  设  T  是闭算子. 因  E, E_{1}  都是巴拿赫空间, 于是  E \oplus E_{1}  也是
巴拿赫空间, 其中的范数由等式(8)定义.
由于  G_{T}  是  E \oplus E_{1}  的闭子空间, 故  G_{T}  也是巴拿赫空间(参见P81).
定义算子

\widetilde{T}: G_{T} \rightarrow E, \quad \text { 满足 } \widetilde{T}(x, T x)=x

首先证明,  \widetilde{T}  是  G_{T}  到  E  上的双映射.  \widetilde{T}  显然是线性的且为满映射.
今设  \widetilde{T}(x, T x)=\theta .  由定义可知,  x=\theta .  于是  T x=\theta,  故  (x, T x)=(\theta, \theta) . 
因此  \tilde{T}  是单映射.

% --------------------------------------


% 整页 =================================
因为

\|\widetilde{T}\{(x, T x)\}\|=\|x\| \leq\|x\|+\|T x\|=\|(x, T x)\|

所以,  \widetilde{T}  有界.
由定理2.2,  \widetilde{T}  存在有界逆算子  \widetilde{T}^{-1} .
于是对  \forall x \in E,  由

\begin{array}{c}
(x, T x)=\widetilde{T}^{-1} x \\
\Longrightarrow \quad\|T x\| \leq\|(x, T x)\|=\left\|\widetilde{T}^{-1} x\right\| \leq\left\|\widetilde{T}^{-1}\right\| \cdot\|x\|
\end{array}

因此  T  有界.

% --------------------------------------


% 整页 =================================
" \Rightarrow ": \quad  设  T  有界.  \forall(x, y) \in \overline{G_{T}},  存在  \left(x_{n}, T x_{n}\right) \in G_{T}  使得

\left(x_{n}, T x_{n}\right) \rightarrow(x, y) \quad(\text { 当 } n \rightarrow \infty) .

于是

x_{n} \rightarrow x, \quad T x_{n} \rightarrow y

因  T  有界, 故  T x_{n} \rightarrow T x .  于是  y=T x .  由此可知,  (x, y) \in G_{T} . G_{T}  闭, 即  T  为闭算子. 证毕.
注: (1) 定理2.4相当重要, 因为它将判断定义在巴拿赫空间上的线性算
子是否有界转化为判断该算子是否为闭的. 这在不少情况下很方便.
(2) 但如果闭算子的定义域仅仅是巴拿赫空间的一个子空间, 则它不一
定有界, 试观察下面实例。

% --------------------------------------


% 整页 =================================
例  \quad  微分算子  T=\frac{d}{\mathrm{~d} t}: D_{T} \rightarrow E_{1},  其中  D_{T} \equiv C^{1}[a, b] \subset E=C[a, b] 
而  E_{1}=C[a, b] .  则  T  是无界闭算子.
1)  T  是闭算子.
实际上, 设  \left\{x_{n}\right\} \subset C^{1}[a, b]  且在  C[a, b]  中  \left\{x_{n}\right\} \rightarrow x,\left\{T x_{n}\right\} \rightarrow y  同时成立.
第二个极限实际上是指函数列  \left\{x_{n}^{\prime}\right\} \rightarrow y \quad(n \rightarrow \infty) .  即函数列  \left\{x_{n}\right\}  以及
函数列  \left\{x_{n}^{\prime}\right\}  分别一致收签于  x(t)  及  y(t) .
由数学分析可知,  x(t)  具有连续导数  x^{\prime}(t)  且  x^{\prime}(t)=y(t) .
因此  x \in C^{1}[a, b]  且  y=T x .  由定理2.3可知,  T  是闭算子.
2) 但前面已经证明  T  无界(参见P134 例7).
注: 若  E=C^{1}[a, b]  呢  ? 

% --------------------------------------


% 整页 =================================


% --------------------------------------


% 整页 =================================


% --------------------------------------


% 整页 =================================


% --------------------------------------


% 整页 =================================


% --------------------------------------


% 整页 =================================


% --------------------------------------


% 整页 =================================


% --------------------------------------


% 整页 =================================


% --------------------------------------

















































































